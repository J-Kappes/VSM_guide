\chapter*{BIBLIOGRAPHY} \label{BIBLIOGRAPHY}
\addcontentsline{toc}{chapter}{BIBLIOGRAPHY}
compiled by John Waters
\hrule
For the theoretical background to the VSM and for an account of Beer's work in Chile, refer to:
\begin{itemize}
	\item Brain of the Firm 2/e
	Stafford Beer (John Wiley, 1981)
	A largely neurocybernetic development of the VSM and (introduced in the 2nd edition) an account of Beer's work in Chile.
\end{itemize}

The VSM is further developed, from a different perspective and with refinement/standardization of the terminology and diagrammatic conventions, in:
\begin{itemize}
	\item The Heart of Enterprise
	Stafford Beer (John Wiley, 1979)
	A companion volume to "Brain of the Firm" which develops and illuminates the VSM from additional perspectives.
	\item Diagnosing the System for Organizations
	Stafford Beer (John Wiley, 1985)
	A guide (aimed at managers) to applying the VSM.
	\item The Viable Systems Model: Interpretations and Applications of Stafford Beer's VSM
	ed Raul Espejo \& Roger Harnden (John Wiley, 1989)
	A diverse collection of papers dealing with many different aspects of the VSM and the foundations on which it is built. Case studies, critical reinterpretations and alternative perspectives.
\end{itemize}
Further useful and illuminating material can be found in the following:
\begin{itemize}
	\item Platform for Change
	Stafford Beer (John Wiley, 1975)
	A book which, while not explicitly mentioning the Viable System Model, deals with many issues fundamental to the continuing viability of human systems.
	
	\item Designing Freedom
	Stafford Beer (John Wiley, 1974)
	A series of six lectures, originally broadcast on Canadian radio, briefly covering some of the same ground as "Platform for Change".
	
	\item Decision and Control
	Stafford Beer (John Wiley, 1966)
	A cybernetic treatment of Operations Research.
	
	\item A Complexity Approach to Sustainability Theory and Application
	Angela Espinosa \& Jon Walker (Imperial College Press, 2011)
	
	\item An Introduction to Cybernetics
	W. Ross Ashby (Chapman and Hall, 1956)
	A thorough and accessible introduction to many aspects of the theoretical foundations upon which the VSM was built. In particular this book introduces, develops and justifies the Law of Requisite Variety. It is now available in PDF format.
	
	\item Design for a Brain 2/e
	W. Ross Ashby (Chapman and Hall, 1960)
	This book illustrates a number of very important concepts, including homeostasis and ultrastability.
	
\end{itemize}