\chapter{Preface}

\section*{Beginnings ...}
By about 1985 it had become clear that we were running out of options.

Co-operatives in the UK were becoming large and the techniques which had served us well while numbers were small were clearly not working.

I was working for a co-operative called Suma and at this time we had reached around 35 workers. We still expected to base our decision making and organisation around a once a week management meeting which had become a shambles due to the numbers involved ...

But what could we do?

Politically we found methods based on obedience and authority unacceptable (and besides the rumours from the traditional world were that these ideas were being questioned ...)

But the lack of structure was getting very frustrating and everyone was becoming aware that structurelessness could result in just the sort of unjust working practices which we were trying to avoid.

Many co-operatives in the UK were faced with similar problems and some had decided to adopt traditional solutions. In several instances, they were turning to elected managers, and the majority of members were put under their control.

In this climate I wrote to a man called Stafford Beer who had created the Viable Systems Model, and began to apply his ideas to co-operatives.

In my initial discussions with Beer, I was looking for the answer to two questions:
\begin{samepage}
\begin{itemize}
	\item Was the Viable Systems Model an appropriate vehicle for looking at problems in co-operatives?
	
	\item Did the Viable Systems Model at any point require the use of authority and obedience?
	
\end{itemize}
\end{samepage}

Beer was completely clear on both these issues: Yes, the VSM was powerful enough to deal with the kind of problem I was looking at, and No, the VSM did not require hierarchical management techniques in any shape or form.

In fact, Beer himself felt that while the use of managerial authority appeared to be an easy way of dealing with organisational problems in the short term, it is really a very crude solution, and that the most appropriate way to create an efficient business is to give everyone as much autonomy as possible.

I found his ideas fresh and exciting and entirely compatible with the basic human values which underlie co-operative working.

The application at Suma (which is described in one of the Case Studies) demonstrates that an effective organisational structure can be based upon individual freedom, and that authoritarian management is not the only alternative.

\section*{Credibility ...}
Although I had read at some length about the Viable Systems Model and found the basic approach fascinating, my main concern was with its credibility.

There was no point undertaking what appeared to be a very lengthy and difficult study if the VSM was an academically interesting idea with very little applicability.

I had hoped that I could go to a small business and say \textit{OK, what did it do for you?} but this proved to be impossible. Most of the applications were in extremely large companies and most of them were outside the UK.

Beer himself had no hesitation about the use of the VSM. He had used it in consultancies for four decades and had been able to demonstrate increases in efficiency of between 30 and 60\%.

His list of applications includes the steel industry, textile manufacture, shipbuilders, paper manufacturers, insurance companies, banks, transportation, education and a plethora of small businesses including both manufacturing and retailing.

There is a list of applications in the back of one of his books and some case studies are described. Many of these have been collected in "The Viable System Model - Interpretations and Applications of Stafford Beer's VSM" by Espejo and Harnden (1988).

The most remarkable story involves the application of the VSM in Chile in 1971. Beer was invited by President Allende to study the entire social economy and to make changes to the existing systems of organisation. In the 18 months before Pinochet came to power in a bloody coup, Beer managed to organise something like 75\% of the Chilean economy into a single integrated information system, and to demonstrate remarkable increases in efficiency.

The work in Chile remains an extra-ordinary experiment and a source of inspiration to those of us who don't think that since the collapse of the Eastern European centralist systems referred to as Communism, that the Free Market is the only alternative.

All of these accounts and subsequent meetings with other consultants who use the VSM led me to believe that the VSM had the necessary credibility to warrant a thorough study.

Five years later, I am in no doubt about the usefulness of the model. The theory is difficult in places and the overall conception seems strange when you begin, but the emerging understanding of the mechanics of viability in an enterprise seems to give the VSM an unprecedented power to find out how things actually function and to pin-point areas which need attention.

\section*{How the rest of this Manual is Organised}
\textbf{Section 0: \hyperref[CYBERNETIC EYES]{Cybernetic Eyes}}

An introduction to VSM principles.

\textbf{Section 1: \hyperref[THE QUICK GUIDE TO THE VSM]{The Quick Guide to the VSM}}

An overview of the entire process in 6 pages. This is designed to introduce most of the ideas and show how they will be applied. It is for those of you who don't have the time to undertake a complete diagnosis.

\textbf{Section 2: \hyperref[CASE STUDIES]{Case Studies}}

A description of the use of the VSM in co-operatives of various sizes.

\textbf{Section 3: \hyperref[PRELIMINARY DIAGNOSIS]{Preliminary Diagnosis}}

How to identify the various parts of your own organisation which are crucial to viability and draw a large VSM diagram of your organisational structure. This process points out any gross shortcomings in your structure and any existing parts which are not involved in viability.

\textbf{Section 4: \hyperref[DESIGNING AUTONOMY]{Designing Autonomy}}

How to maximise the degree of autonomy in the Operational parts of your organisation. How to ensure they hang together in an integrated, coherent form, and don't threaten overall viability.

\textbf{Section 5: \hyperref[BALANCING THE INTERNAL ENVIRONMENT]{The Internal Balance}}

How to balance the Operational parts of the internal environment with the parts needed to co-ordinate and optimise. How to avoid authoritarian management.

\textbf{Section 6: \hyperref[INFORMATION SYSTEMS]{Information Systems}}

How to design information gathering system which work with daily information. How to use these to generate alerting signals and not loads of useless printouts. How to design information systems to underwrite individual and departmental autonomy.

\textbf{Section 7: \hyperref[BALANCE WITH THE EXTERNAL ENVIRONMENT]{Balance with the Environment}}

How to design of the Future Planning systems which must develop strategies in the context of a constantly changing environmental. This must be in balance with both the external environment (markets, etc.) and the internal environment (capabilities of the Operation ...)

\textbf{Section 8: \hyperref[DESIGNING POLICY SYSTEMS]{Policy Systems}}

How to involve everyone in the Policy decisions without involving huge amounts of time in meetings.

\textbf{Section 9: \hyperref[THE WHOLE SYSTEM]{The Whole System}}

How the whole thing works together.

\textbf{Section 10: \hyperref[APPLICATION TO FEDERATIONS]{Application of the VSM to Federations}}

\section*{Introductory Notes}
\begin{enumerate}
  \item This is not a complete theoretical study of the Viable Systems Model. A complete study of the VSM is a huge undertaking, and in my experience has put many people off getting to the stage where they can actually apply it. I have included only those aspects of the VSM which have proved to be of most value in my own practical applications. Many of Beer's Axioms and Principles have been translated into everyday language, hopefully without loss of precision.

  \item For the theoretical background to the VSM and for an account of Beer's work in Chile, refer to \textit{Brain of the Firm} 2/e, Stafford Beer (John Wiley, 1981). (See \hyperref[BIBLIOGRAPHY]{bibliography})

  \item The nomenclature has mostly been introduced. The words which apply directly to the VSM begin with capital letters (\textbf{O}peration ... \textbf{M}etasystem ...), other key concepts such as autonomy and synergy are used in the conventional manner and do not warrant capital letters. The Operation is described as being composed of Operational \textit{units} or Operational \textit{elements}. There is no difference between these two terms. Similarly the internal environment is sometimes called the \textit{Inside and Now}.

  \item Several people who were extremely helpful in getting this pack to its original form, an A4 booklet, were \textit{ipso facto} helpful in getting it into its current, unanticipated form. So, thanks are due to Pam Seanor and Toby Johnson for reading the original proofs, to CAG and CRU for suggestions on its approach, and to Dennis Adams, Steve Hey and Roger Harnden of Liverpool Poly for technical advice and for some of the examples. Special thanks are due to Roger Kelly of the {Centre for Alternative Technology}\footnote{\href{http://www.cat.org.uk/}{http://www.cat.org.uk/}} who agreed to test the original version of the pack and who got far enough to see it was far too theoretical and that it required a complete re-write in order to be of use to the average Social Economy organisation. And finally to Stafford Beer himself who has been unfailingly generous with his time and encouragement, especially in the more difficult stages of the work.

\end{enumerate}

\noindent\rule{\textwidth}{0.5pt}
\hyperref[BIBLIOGRAPHY]{\textbf{Bibliography}} (compiled by John Waters)

\hyperref[LINKS]{\textbf{Links}} (compiled by John Waters)

The \nth{2} edition of \textit{The VSM Guide} can be found here\footnote{\href{http://www.esrad.org.uk/resources/vsmg_2.2/}{http://www.esrad.org.uk/resources/vsmg\_2.2/} (down as of publication)}. There is also a PDF version\footnote{\href{http://www.esrad.org.uk/resources/vsmg_2.2/pdf/vsmg_2_2.pdf}{http://www.esrad.org.uk/resources/vsmg\_2.2/pdf/vsmg\_2\_2.pdf} (down as of publication)} (583kb) of the \nth{2} edition which may more useful if you want to print hard copy or keep it as hand reference in a single document. The contents of the \nth{2} edition are substantially the same as those of the \nth{3}.

\noindent\fbox{
	\parbox{\textwidth}{%
		\centering
		\parbox{\textwidth - 6mm}{%%
			\vspace{3mm}
\textbf{COPYING \& REDISTRIBUTION}\\\\
Copyright © 1998 by Jon Walker. You are welcome to copy the \textit{The VSM Guide} (either edition), in whole or in part, and to make copies available to others, \textbf{if you acknowledge both its source and its authorship}. Although freely distributable and presented for unrestricted use, \textit{The VSM Guide} is not in the Public Domain.
			\vspace{3mm}
		}%%
	}%
}