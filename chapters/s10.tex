\chapter{APPLICATION TO FEDERATIONS}
  \label{APPLICATION TO FEDERATIONS}
In this Section the VSM will be applied to the question of how Social Economy enterprises may join together to form a mutually beneficial federation.

The Section begins with a number of case studies involving attempts by co-operatives in the UK to federate, and contrasts these with the example of the \href{https://www.mondragon-corporation.com/en/}{Mondragon} co-operatives in northern Spain.

Some conclusions are drawn in terms of the Viable Systems Model.

The question of design is then considered by using the preliminary work of Stafford Beer on a federation of Nation States in South America.

The Section concludes with some design consideration which will supplement the step by step methodology developed in this pack, for the specific case of federations of Social Economy enterprises.

\section*{The Federation of Northern Wholefood Collectives}
The FNWC began in the mid 1970's as a result of Reg Tayler's agreement with the wholefood shops in the north of England to buy their goods through him and thus to establish a co-operative warehouse (Suma).

Once Suma was in business the number of wholefood shops in the region began to grow and over a two year period around 60 retail co-operatives were set up, all buying from Suma.

The FNWC was the umbrella for all of these co-ops, and monthly meeting were held to discuss the kind of goods and services which Suma should offer, to welcome the new shops which had been set up and to attempt to direct the future course of the Federation.

The rapid growth and success of the Federation led to a mood of optimism, and it was not uncommon for people to express the feeling that co-operatives had finally managed to make a real impact on society. At that time co-operatives were handling around 60\% of the wholefood business in the UK (according to The Grocer trade magazine).

The Federation began to put a levy on all its sales and managed to collect several thousand pounds over a period of a few months.

The possibility of a number of regional warehouses was raised, as the number and geographical spread of the shops was growing. Clearly a local warehouse would provide the best service, but the economies of scale militated in favour of a large central warehouse, and Suma had just moved to a large new premises and needed the support of all the shops in the federation.

It was around this time that the FNWC began to decline. This was partly because Suma had become well established and thus one of the aims of the Federation had been achieved, but also because there seemed to be a lack of direction as to where to go next.

The money which had been raised all went to pay the debts of a restaurant in York, and thus the opportunity to invest in strengthening the FNWC was lost.

Over the next few years attendance at the meetings began to dwindle and it became grudgingly accepted that the loan fund was not operating successfully. After a couple of recipients got into financial trouble and were unable to complete the repayments, the loan fund was abandoned.

Currently, there is little left of the Federation. Suma has withdrawn its discount to co-operatives, and generally trade is the main area of contact.

In retrospect the collapse of the Federation was due to:

\begin{itemize}
  \item Suma's growth and prosperity thus making some aspects of the Federation unnecessary.

  \item Lack of vision.

  \item Lack of the necessary skills to make it work.

\end{itemize}

\section*{The Federation of Wholefood Warehouses in the UK}
In the mid 1980s, there was a series of meetings to establish a Federation of the 6 wholefood warehouses which were in business in the UK. The supposed advantages were:

\begin{itemize}
  \item The capacity to offer a local delivery service to any shop in the UK.

  \item A coherent marketing policy.

  \item Joint buying thus giving better prices and saving the labour involved in having 6 businesses all buying the same commodities.

  \item The ability to produce and market a series of own label products.

  \item Sharing expertise.

\end{itemize}

Despite the apparent good sense of all this, Suma decided not to join. On the information which was available to the co-operative it looked as if Suma would pay most of the cost of running the Federation and get very little back in return, The problem was that Suma's turnover was as much as the other wholesalers put together, and thus the buying price advantages had already been achieved. There was also a problem with delivery areas. Suma was in competition with some of the other warehouses in certain areas, and this was never resolved. In one particular area, Suma was geographically close but one of the other warehouses had a long term personal relationship with a local distributor.

Consequently the Federation was formed without Suma, and thus was only able to offer its services to half the country. The ideal of a national federation of warehouse was rendered null and void.

Soon after it was established, one of the founder co-operatives went out of business, further weakening the federation. Own-label products were produced, but some of these were in direct competition with Suma's products. Centralised buying and marketing never really happened as the member co-operatives were loath to give up any autonomy.

Currently, two of the warehouses in Bristol are in the process of merging as they were operating within 12 miles of each other. And in Scotland two new warehouses have opened in order to provide a more local delivery service over a large and scattered market.

Generally all of these developments happen in a piecemeal fashion, and there is little sense of planning the distribution of wholefoods throughout the UK in a coherent manner.

There are still problems with the "right" to distribution areas. Suma has recently extended its weekly deliveries into South Wales which is close to the Bristol warehouses. And one of the Scottish warehouses is actively selling in Northern Ireland which has traditionally been part of Suma's area.

Without doubt the attempt to get the co-operative wholefood warehouses in the UK to federate and act a cohesive system has failed.

\section*{Federations of Mondragon Co-operatives}

\subsection*{Introduction}
Mondragon is a small town in the Basque region of Spain which has been the location for an extra-ordinary experiment. During the last 50 years the region has been completely changed by the development of 173 separate co-operatives which now employ over 22,000 people.

Everything about Mondragon is impressive: they turn over about three billion dollars; they re-invest vast sums of money in order to keep their buildings and machinery in excellent condition; they have their own research and development labs which are developing state-of-the-art computer and robotic systems They have their own schools, bank and social security system; and they are completely dedicated to the ideal of co-operation.

There is little doubt that the success of the Mondragon co-operatives is to some degree the result of their ability to form alliances and work together at both the geographical and trade-sector level.

The present diagnosis is concerned with the application of the VSM to the way in which they organise separate co-operatives into a coherent system at the Sector Level.

\subsection*{Autonomy and Cohesion}
Everyone I spoke to was adamant that the member co-operatives - which constitute the System 1 or Operation - are completely autonomous. There were accounts of how the members of a small co-operative which makes heaters refused to bow to a sector decision and merge with a larger co-op. The Metasystem was informed of this decision and its only recourse was persuasion. The relevant personnel returned to the small co-op, presented the arguments for the merger and eventually got agreement. However, the small co-operative could have refused.

In VSM terms, the complete autonomy given to each co-operative may be a little excessive: a mechanism should exist which required the merger in the interest of system synergy. However, as everyone I spoke to at Mondragon seems more concerned with the good of the whole than an individual co-op, the system currently seems to work. It was also mentioned that the small co-operative faced exclusion from the group if it failed to respond to the arguments for merger.

So while there seen to be no formal systems to ensure autonomy has to become subject to system cohesion, the social pressures are enormous.

.

\subsection*{Sectoral Organisation}

\subsection*{System 1: Operation}
The 173 co-operatives in Mondragon are currently re-organising from a regional collaboration to a Sectoral base. They are quite clear that in order to compete in a unified Europe, they need to concentrate on the synergy (their words ...) between enterprises in the same Sector.

For this study, the System-in-Focus will be a collaboration of typically 8 co-operatives in a single sector. Mondragon has several sectors dealing with domestic appliances, castings, food, service industries and so on. The principles on which the collaborations work are identical, although some of the details are different.

In general, it will be adequate to talk non-specifically about viability within a Sector, although specific examples will be given.

Within each co-operative, there seems no doubt that the organisational systems cover all of the aspects of VSM diagnosis that I have been involved with. The system of autonomous work groups and daily measurement of performance indicators (although not the concept of statistical filtration and thus automatically generated algedonics) has been up and running for a decade, and they find it leads to greater motivation, a more enjoyable work environment, greater productivity and higher standards.

\subsubsection*{SUMMARY: SYSTEM 1}
System 1 consists of about 8 autonomous co-operatives.

Each co-operative functions within the same Sector.

Each co-operative exhibits the essential property of being a Viable System embedded in the whole.

The Mondragon co-operative culture is focused on the whole group of co-ops, rather than the single enterprise. This provides the basis for federation.

\subsection*{System 2: Stability}
Perhaps the most astonishing aspect of Mondragon is the systems it employs to ensure all the co-operatives avoid instabilities. The word solidarity comes up time after time, and it's difficult not to be impressed by the feeling that all 22,000 Mondragon worker-members are working together. The concept of one co-operative benefiting at the expense of another seems difficult for them to grasp, and usually my questions about conflict of interests were met with surprise.

However, the Eroski Food Group are building HyperMarkets which are taking trade from the smaller shops in the group. This seemed a potentially unstable situation (different co-operatives fighting for the same market) and looked problematical to me. However, at Mondragon, they don't see a problem for the following reasons:

\begin{enumerate}
  \item If one co-operative in a sector loses money, it is supported by the others. Historically most co-operatives have needed support at one time or another from the group, and this is seen as a system of mutual support. It means that there is absolutely no problem with one shop gaining at the expense of another as long as the combined profits of the group are growing. This seems to generate the view that it's the Sector that matters, rather than the individual co-op.

  \item All wages are the same throughout the Sector.

  \item All profit sharing is made on a Sectoral basis.

\end{enumerate}

Thus, if a small shop is loosing trade due a new hypermarket, it is quite possible that the worker-members in the small shop will see the profits of their own co-operative fall but get higher remuneration at the end of the year as the profits of the Sector will rise.

(Compare this situation with the UK wholefood warehouses. As each co-operative had its own wages system and there was no suggestion of mutual financial support, the interactions were unstable: there was conflict of interests due to competition for a limited market. The lack of any sort of System 2 to deal with this meant that the federation was doomed.)

The Mondragon co-ops, who share wage scales and profits throughout the Sector, have managed to dissolve all these problems. They see this as solidarity: in Viable System Model terms, it is an extremely effective System 2 which ensures the member co-operatives can look beyond the survival of their individual co-operatives and concentrate on the development of the Sector.

System 2 is a prerequisite for the articulation of an integrated Viable System.

\subsection*{System 3: Synergy \& Optimisation}
Each co-operative in the Sector sends its General Manager to a Sectoral General Management meeting. Sector General Management is described as stimulating the "co-ordinated, harmonious joint development of the co-operatives incorporated in the Group" Clearly this is a System 3 function.

Some examples of how they do this follow:

\begin{enumerate}
  \item Joint education, research and development. This includes education at all levels throughout the co-operatives. (Beyond the means of one, possible for the Sector).

  \item Optimised product ranges, trade marks etc.

  \item Being able to offer a customer a complete service. (For example there are several co-operatives which make castings: each now specialises in one aspect, the Sector can offer the complete range. This also required new co-operatives to be established so that the Sector could offer a complete service.)

  \item Centralised buying, publicity and marketing.

  \item Transfer of technology and expertise between co-ops.

  \item Inter-co-operative trading: All the manufacturing co-operatives use Mondragon machine tools and control gear.

\end{enumerate}

Generally it's accepted that there are enormous advantages to collaboration. At the level of the entire Mondragon organisation there are bodies called "superstructural". These include a research and development institute, a Social Security system, a Bank, a Technical College, and a training centre for co-operative and management skills. None of the services which these offer would have been possible without collaboration, and it is certain that Mondragon would not have developed so successfully if the member co-operatives had grown as isolated businesses.

The systems they employ to encourage Sector synergy involve meetings of the general managers of the various co-operatives in a Sector and the appointment of a Sector Manager. It should be noted that the power of decision is delegated up through the usual Mondragon system of General assembly, and that the member co-operatives have to agree to the recommendations.

\textbf{Clearly they take advantage of every opportunity to deal with Sector synergy. System 3 is alive and working well.}

\subsection*{System 4: Future Planning}
The existence of the Sector Management bodies enables the Managers to examine the environment in which they exist and to plan accordingly. This System 4 function is performed with characteristic Mondragon excellence.

The FAGOR group, which manufactures a complete range of consumer products, industrial components, and engineering equipment employs 8,000 worker-members with sales of around one billion US dollars. They produce 30\% of the Spanish home appliance market. The 1989 annual report lists some of the System 4 planning activities which were realised.

\begin{enumerate}
  \item Purchase of a company manufacturing refrigerators and electrical boilers.

  \item Purchase of a company manufacturing car components.

  \item Establishing links with specific European and North American companies for joint R\&D projects in domestic appliances and automation.

  \item Launch of a manufacturing plant in Mexico.

Financing advanced production technologies, leading to highly flexible manufacturing processes.

  \item Allocation of \$21.2 million for R\&D activity.

  \item Collaboration with European universities and research centres in ventures like the European Space Project.

\end{enumerate}

The report also outlines their basic commitment to System 4 Activity. For example: "The new economic situation requires a high degree of innovation."

Clearly they are performing the System 4 activities admirably: they are in touch with their environment and are planning at the Sector level to adapt to future threats and opportunities.

The Mondragon philosophy requires an unprecedented commitment from its members, and part of this is realised by the continuous re-investment of its profits both in R\&D and in improvements to plant and buildings. It also results in continuous training and re-training.

In terms of the mechanisms required to accept the plans, Mondragon continues its emphasis on democracy. The plans are made by the Managers, but have to be passed by the Board of Directors. The system, they feel, is slower than in a traditional company but has the advantage that everyone agrees once a decision is made.

However, despite their perceived tardiness, the results of System 4 activity cannot be seen as inadequate in any sense.

\subsection*{System 5: Policy}
Policy is determined by a General Assembly and is described as "The Will of the Members."

Throughout Mondragon, attempts are made to involve the maximum number of members at the policy level. Currently the entire group is debating the wage differentials and the decision will be made at a meeting of over 1,000 people.

At the Sector level, the General Assembly is composed of the members who are on the Board of Directors of every co-operative in the Sector together with the Audit groups and the Managers. Their tasks are described as "to approve the Groups general policy, approve the budget, and to modify organisational rules"

The Eroski group consists of both the 1928 worker members who run the shops and warehouses and the 152,000 consumer members. In this case the General Assembly is composed of 250 delegates from worker-members and 250 delegates from the consumers. In order to ensure everyone is informed of the issues, there are five preliminary meeting before the General Assembly.

The large number of notice boards throughout the factories that I visited, together with the interest that was shown in issues at all levels (currently wage differentials and restructuring to emphasise the Sectoral rather than the Regional groupings) indicated that everyone has an interest in policy at all levels and the means to express themselves.

\subsection*{SUMMARY - SYSTEM 5}
At the Sectoral Level policy is described as "The Will of the Members".

Everyone in Mondragon seems to be interested in policy decisions and there are mechanisms to express this.

Some reservations must be made on the year gap between meetings. From my experiences at Suma it seems clear that lots of policy issues are happening on a weekly time scale, and that to restrict the members by making a contribution once a year has to be seen as a limitation on their involvement in policy matters.

\section*{Conclusions: The VSM and Federations in the UK and N. Spain}
The first point to make is that the Mondragon co-operatives developed in an entirely different setting to their UK counterparts, and there is a common viewpoint that these differences make any comparison invalid. However, as the VSM provides a different language to discuss these issues, and is concerned with structure rather than with cultural background, these objections do not apply.

As a general point it should be noted that the Mondragon co-operatives were directed by Arizmendi, and his overview (or metasystemic view) was without doubt one of the main driving forces behind the growth of the co-operatives. For example, the establishment of the bank would never have happened as the co-operatives which existed at that time were preoccupied with their own growth, and had to be persuaded that diversion of resources to a bank was a good idea.

Without Arizmendi's vision, its likely that the co-operatives would have developed in isolation.

It's also worth noting that Mondragon offered far more than employment. In post war Spain there was no social security, no health care and no pensions. The vision Arizmendi had of Mondragon was a complete way of life and thus he could ask much more of the people.

In the UK co-operatives exist in a culture where there are other jobs, and a well established system of social security.

\subsection*{System 1: The Member Co-operatives}
In all cases the Operational units of the federation are viable autonomous businesses and from this point of view there is no difference between the Spanish and UK co-operatives.

The basic requirement is that they are prepared to give up some of their autonomy in the interests of coming together as a coherent larger whole, and here the differences begin.

In Mondragon it is obvious that commercial success has resulted from the synergy which comes from various co-operatives working together In the UK the track record is very different: federations have achieved virtually nothing, and co-operatives currently work in isolation with only loose trading links and occasional conferences.

Thus the willingness to divert funds and time into a federation, and to accept limits on autonomy in the interests of the greater whole is vastly different in the two countries.

This may be the fundamental difference, and the basis behind the differences in the Metasystem which are discussed in the following pages.

\subsection*{Allocation of Resources}
The Mondragon co-operatives allocate enormous amounts of time and effort to their Metasystemic activities. Each Sector has its managers and assemblies, and there is continuous investment in services which can be offered to the whole group. These include:

\begin{itemize}
  \item Banking

  \item Research and development

  \item Financial advice

  \item Business rescue teams

  \item Preferential loans

  \item Technical advice

  \item Training institutes

\end{itemize}

In the UK co-operatives seem to be loath to allocate any resources whatsoever into activities of this kind. The umbrella organisation for worker co-operatives (ICOM) has had to find ways of financing itself as the membership fees were insufficient, and in general co-operatives can find very little reason to consider working together, and consequently are unwilling to divert resources into Metasystemic federal activity.

In the UK none of the services listed above have been funded by member co-ops. If they exist at all they have been set up independently, and then offered to co-operatives for a fee.

The conclusion is clear: the Mondragon co-operatives have allocated vast amounts of time and money into the creation of a Metasystem charged with "co-ordinated harmonious joint development". In the UK there is no such investment and consequently no Metasystem and no federal activity.

\subsection*{System 2 - Stability}
Mondragon stability systems may be listed as follows:

\begin{enumerate}
  \item Mutual financial support for all member co-operatives: This ensures that loss of sales in one area as a consequence of new sales in another is not a problem.

  \item Wages and profit sharing on a Sector Basis.

  \item Guaranteed relocation of staff throughout the Group.

\end{enumerate}

These factors are all entirely absent from the UK co-operative movement, and thus many co-operatives find themselves in a competitive situation. Generally it has been impossible to resolve conflict of interests and the over-riding factor has been the self-interest of the individual co-operatives.

System 2 is a prerequisite of any Viable System and it should be noted that:

\begin{itemize}
  \item Mondragon has an excellent System 2

  \item The UK federations don't.

\end{itemize}

It is also inevitable that part of the job of Mondragon Sector Management will involve scheduling and co-ordination of member co-operative activity and this (again) is System 2 activity.

And (again) as the UK has never put resources into Sector Management, this kind of activity does not occur.

\subsection*{Systems 3, 4 and 5}
The case study on Mondragon went to some length to describe Systems 3 and 4, and following the discussion on System 2 it should be obvious that in the UK the lack of resources allocated to the Metasystem guarantees that no such activity can occur and that therefore the federations cannot be viable.

It should also be noted that The Mondragon co-operatives have also realised the need for the Federal System 3* (audit) channel which monitors the accounts of the member co-operatives and alerts the various federal bodies as and when they deem it necessary.

From VSM perspective this is needed to ensure that System 3 has all the information it needs to deal with its internal environment.

The collapse of one the wholefood warehouses came as a complete surprise, and could possibly have been avoided if the Federation audit channel had been set up.

Had this occurred within a Mondragon Sector there would have been:

\begin{itemize}
  \item warning through the Audit Group and

  \item attempted rescue though the bank's trouble shooting teams.

\end{itemize}

\subsection*{Summary: Preliminary Diagnosis}
From these considerations, it is clear that on undertaking a Preliminary Diagnosis of federations in northern Spain and in the UK, the differences are startling.

Whichever system you consider Mondragon has a clear articulation and several subsystems to carry out its overall task ... and in the UK it is hard to find any Metasystemic activity whatsoever.

Despite the differences in culture, if VSM theory is correct then UK co-operatives must find some way of dealing with instability, optimisation and future planning at the Sector level.

Until this is done there is no possibility that a Federation will exhibit the basic requirements which will render it viable.

It is likely that the methods used in these countries will be different from those employed by the Spanish but nevertheless, it will be essential to find the resources needed to articulate the Metasystem.

It will also be necessary to design its capabilities to ensure they match the requirements of the System 1 co-operatives, but until the need for a Metasystem is accepted this remains nothing but a pipe dream.

\section*{Preliminary Studies on a Federation of Latin American Countries}
During 1988 Beer was asked to undertake a study on the prospect of forming an alliance of democratic Latin-American countries which could challenge the power of US dominated finance.

The political events that followed (ironically caused as Beer notes by response to IMF pressures) ensured that the study was never concluded but preliminary studies were made and are of direct relevance to the present subject.

Clearly the System 1 Operational elements are the member countries and the mission is to provide a united challenge to financial oppression. However such a federation would also offer the possibility of synergistic interactions. Generally the purpose of the alliance may be defined as mutual benefit.

Beer begins with the 6 vertical channels which depict the interactions between the various System 1 units.

\subsection*{1. The Command Channel (Mandatory prohibitions)}
As the purpose of the alliance is mutual benefit, very few restrictions are put on the member countries apart from an agreement not to violate each others interests. This would be the only function of the command channel and would be supervised by the presidents of the countries concerned.

\subsection*{2. Resource bargain (Negotiated prohibitions)}
The resource channel is charged with creating whole system synergy. The amount of resources allocated to deal with these issues would have to be decided by the mutual consent of the presidents concerned. It is likely that given the nature of the alliance the possibilities for synergy would have to be clearly specified before the resources were allocated.

\subsection*{3. System 2. (Stability)}
The design of an appropriate System 2 is crucial. Beer quotes both the UN and the EEC as organisations which have articulated a System 2 which has developed into a "malignant cancer of bureaucracy" The appropriate design of a System 2 for both the Latin American states and the collapsing empire of the Soviet Union remains one of the most exciting challenges for humanity.

\subsection*{4. Interaction of Environments}
Each country shares borders both geographically and in terms of economic alliance because of trade. It shares coastal waters and atmosphere. Also "Latin-Americanism" in business \& tourism. Above all it shares its role in the hegemony that seeks to control it from outside. These factors need to be studied and recommendation made as to the interactions in the noted areas.

\subsection*{5. Interaction of Operational Units}
This channel deals with the practicalities of movement of goods between the countries. Currently it is common for half the population to live in the capital, exporting the country's wealth - with little or no added value. Study of this channel should focus upon the movement of goods between regions, using manufacturing skills in one country to add value to raw products from another, and then moving to another member nation for final processing.

\subsection*{6. System 3* The audit channel}
This channel performs occasional audits of the Operational units at the request of System 3 in order to fill any gaps in its knowledge.

In Uruguay, information gathered on 3* about enhanced grain products led to special deals with Argentina and Brazil, initiated by a vestigial System 3 looking for synergistic interactions.

Beer feels that much could be done in the larger alliance with semiprecious stones, natural spring water, cottage industries in co-operatives and so on.

**

This concludes Beer's discussion of the six vertical channels. His basic strategy is to shift the basis of control away from the central command channel.

The essence is to preserve the delicate balance between the autonomy of the countries in the alliance and the needs of the Metasystem to hold the whole thing together.

**

\subsection*{Systems 2 \& 3: Articulation}
From the previous arguments it is clear that System 3 must be as minimal as possible. After the design of the environmental and Operational interactions, and the agreements for 3* to carry out its audits and surveys, Systems 2 and 3 may be designed to ensure that they can do whatever is needed to stabilise and optimise the member countries in the alliance.

Clearly, for political reasons, this would have to involve minimal interference or the member countries would be likely to withdraw.

\subsection*{System 4: Articulation}
System 4 is the "amalgam of free national spirits within the alliance."

System 4 is charged with dealing with those issues of planning and adaptation which are appropriate to the alliance rather than the member countries.

Beer sees this as the greatest challenge of all. His diagnosis of the debacle in both the East and the West is that:

\begin{enumerate}
  \item they have both massively overdone the use of the central command channel and that

  \item they have both failed to design a proper System 4.

\end{enumerate}

Consequently both are still in fire-fighting mode.

Viability of the Alliance requires it to go beyond this and to design an appropriate System 4 which can plan and adapt to the future.

\section*{Step by Step Application of the VSM to Federations}
Some preliminary work must be done to establish clearly the reasons for proposing the Federation in the first place.

Whatever happens there will be a price to pay in terms of loss of autonomy and allocation of resources.

So be clear about the basic premises:

\begin{itemize}
  \item What are the expected advantages of the Federation?

  \item How will the requirements of the Federation interfere with the member co-operatives?

  \item How will the money be raised to fund Federation activities?

\end{itemize}

Assuming that the participating co-operatives still feel that the advantages of Federation outweigh both the loss of local autonomy and the necessary allocation of resources, then the design of the Federation should begin.

This particular application of the VSM assumes that you have read and understood the VSM pack and have made at least a preliminary study of your own organisation.

\section*{Graphical Overview of Federation}

\subsection*{Identification of Operational Units}
To begin with, draw a large VSM diagram and write the names of the member co-operatives against the S1 Operational units.

Now fill in the environments of the of the member co-ops. This will include their markets, suppliers and so on. It is likely that there will be a large degree of overlap, and so the various environmental amoebas will reflect this.

You can now begin to look at the way the member co-operatives interact in terms of

\begin{itemize}
  \item Environmental overlaps.

  \item Movement of goods between co-ops.

\end{itemize}

This exercise should begin to offer possibilities for optimising these interactions.

For example:

\begin{enumerate}
  \item If three warehouses are all sending trucks to the same region it may be possible to draw some regional lines and thus save transport costs.

  \item If all warehouses are buying from the same suppliers each could agree to specialise on a small section of the market and buy on behalf of the others. This may result in more thorough knowledge of that market area and better deals through larger quantities.

\end{enumerate}

\subsection*{Design of System 2}
From the studies on Mondragon, it is clear that their social perspectives offer a pervasive System 2 to deal with potentially unstable situations.

It seems highly unlikely that co-operatives in the UK (who have a well established history of isolated development) would take seriously suggestions to normalise wages throughout a Sector and for profits from one co-operative to be given to another.

If these mechanisms are unacceptable, mechanisms must be found for ensuring the member co-operatives can work together in a stable manner.

System 2 is seen as a prerequisite for a viable system and presents a major challenge for UK co-operatives.

Its job would be to transform a currently competitive situation into a collaborative one.

It would need to look for the instabilities and conflict of interests which would arise between the member co-operatives and to define ways of dealing with them.

\subsection*{Design of System 3}
In the case of Federations, the emphasis on Operational autonomy and the careful design of the other vertical channels (environments, Operational interactions, stability systems, accountability) ensures that most of the work needed to ensure cohesion of the members of the federation has been taken care of./

This leaves System 3 to deal with its main function of Optimisation.

One of the major issues is bound to be the movement of certain functions from the Operational units to the Metasystem. In the case of the wholefood warehouses, it seems obvious that there is a great deal of synergy in moving the buying function to System 3. This would result in one team doing all the buying for all the warehouses, and looking for ways of maximising price-breaks, and moving goods from one place to another to minimise out-of-stocks.

There should be further advantages in using distributed stocks to minimise stock holding, as each warehouse could restock the others if there was a local surge in sales.

However all of this requires the member co-operatives to relinquish their current complete control of the goods they buy.

An alternative would be to design a communication medium which would allow current buying departments to be continuously in touch with the each other. Thus, Suma may find a good price on apricots and let the others know. Someone else may have an even better price. There may be a glut of apricots in Scotland which needs to be sold quickly so that they don't approach their best before date. And so on. This would involve something like a computer bulletin board which all the buyers use regularly and a system to get immediate attention (a pleasurable algedonic) - "There are only five tons left and going fast - who wants them???"

This version of a distributed System 3 (which is nevertheless Metasystemic as it deals with the whole system) has the advantage of avoiding the relinquishment of local control.

It is possible that if it worked efficiently, local centres of buying excellence would appear and everyone would know that Bill in London gets the best deals in walnuts. It is also possible that one buying department proves to be better than all the other and effectively takes over the buying for the whole group. In this case some agreement would be needed to ensure that these buying didn't feel exploited by the other members of the federation.

\subsection*{Design of System 4}
System 4 has to keep up to date on those aspects of the environment which affect the federation and generate the plans that are needed to continuously adapt to future threats and opportunities.

It may decide that more warehouses need to be opened in areas which are not provided with a local delivery service. It may need to research new markets or new product ranges. It may produce a diversification plan if it feels that existing market areas are on the wane. It may decide to form alliances with other enterprises or take over other companies. It may decide to build its own retail outlets.

All of these are strategies for adapting to the future which affect the entire federation.

The question which must be answered are:

\begin{enumerate}
  \item How should these functions be carried out? How many people for how many days a week?

  \item How many resources should be allocated to System 4?

  \item Does it have adequate connections with the external and internal environments?

  \item Does it have facilities to simulate and predict?

\end{enumerate}

\subsection*{Design of System 5}
Clearly, System 5 deals with policy for the entire federation and must embody the views of everyone involved.

This generates some interesting challenges assuming that several hundred people could be involved, perhaps in several countries.

The Mondragon solution is a yearly meeting and several preparatory meetings.

My feeling is that policy requires more regular input, although perhaps less than the weekly meetings held by Suma.

Again the possibility of innovative design using computer communications exist. It may be possible to have continuous debates on policy issues like the debates which are currently conducted on global computer networks. Anyone can log-on at any time, read the previous contributions, and then add her own thoughts.

Such a system could only serve as a preamble to a membership vote and suffers from the lack of face to face discussion, however it would enable views to be aired. It also has the advantage of being available to everyone on the network regardless of separation.