\chapter{INFORMATION SYSTEMS}
  \label{INFORMATION SYSTEMS}
%this section whole section is boxed int
In this Section, the importance of information systems will be discussed, and techniques will be developed to ensure that a thorough system is developed to make sure the various parts of your organisation have the information they need to operate properly.

It cannot be stressed too strongly that a good information system is an alternative to authoritarian management - it can render unnecessary organisational techniques based upon authority and obedience.

If the appropriate information is measured and feed-back is used to modify the way people work, then many organisational issues take care of themselves.

This Section covers real time information systems based on performance indicators, self-assesment using these indicators, and the generation of the alerting signals called algedonics.

\section*{Information Systems: The VSM approach}
If you look at the Inside and Now of your VSM diagram there are several areas in which information is crucial to good design:

\begin{itemize}
  \item The Operational units must be accountable - they must find measures of what they do and make sure the appropriate information gets to System 3.

  \item System 3* does audits and surveys - it is an information service to System 3, looking at whatever is of relevance.

  \item System 3 must have a thorough model of all that it needs to know about the goings-on within the entire complex of interacting Operational units. Otherwise it will be making decisions in ignorance.

  \item An information system must be capable of generating alerting signals so that the organisation can find out that something has gone badly wrong as soon as it happens. These signals are referred to as "algedonics".

  \item System 4 is charged with adapting to environmental change. It will need information about the external environment so it can produce strategies. It will also need a good model of the internal capabilities so it knows what tools it has at its disposal.

\end{itemize}

All of these factors require thorough information systems.

Traditionally information systems within a business are primarily concerned with financial information. They generally involve historical figures, so that after the monthly figures are produced someone may say "We have just realised that the business lost money last month because of something that happened in Factory 27" Which is, of course, too late.

The other aspect of traditional information systems which is superseded in VSM theory is the production of huge print outs from a data base, most of which are never used.

Central to the VSM approach is the production of only what is important. If information says "All seems to be going as usual" then nothing needs be done. Consequently, there is no point in printing the report.

The information systems used in the VSM are fundamentally different from traditional systems in that:

\begin{itemize}
  \item they are based upon performance indicators which measure whatever is important within each Operational unit, and not just financial information.

  \item they are based upon daily measurement so that problems can be identified the same day.

\end{itemize}

\section*{Closing the Loop}
The overall principle is clear - closed loops work and open loops don't.

In traditional businesses the loop is closed in a number of ways: Work hard - get paid - keep the job - satisfy the boss - work harder and so on. The loops are closed with reward systems involving money (and sometimes job satisfaction) and with punishment systems involving the fear of reprimand and eventually losing the job.

All of this works as long as the monitoring systems are adequate enough to keep an eye on the work force most of this time. The problem is the schism that emerges between the motivation of the manager (as much work for as little money as possible thus maximising profits) and the work-force (the opposite).

\section*{Example 1: Suma}
The following example is taken from some of the experimental work at Suma:

\begin{enumerate}
  \item \textbf{The Operational Unit.} This is the pre-packing department which is given a budget and autonomy and expected to do its job properly. It must therefore be accountable to the rest of Suma. That is - it must be able to demonstrate that everything is proceeding in a satisfactory manner.

  \item \textbf{Performance Indicators.} It was agreed with the rest of Suma that the factors which must be measured are:

  \begin{itemize}
    \item Productivity (number of bags per person per day)

    \item Wastage (as a percentage of bulk weight)

    \item Number of Out of Stocks (number of stock lines not available due to packing problems) Stock Holding (value of goods bagged down)

    \item Morale (subjective happiness rating of people doing the packing - the difference between morning rating and evening.)

  \end{itemize}Between them, these indicators provide a complete picture of everything which goes on within the department. Any one of these indicators can identify a problem - for example productivity may be fine, wastage might be improving, the stock holding may be wonderful, but the out-of-stocks may be disastrous. Examination may show that the department is concentrating on efficient long runs, but ignoring the requirement to keep most of the pre-pack lines in stock.

Conversely, out-of-stocks may be fine, but this might be at the expense of productivity. (Lots of inefficient short runs ... )

You can only discharge accountability by monitoring all the indicators.

  \item \textbf{Algedonics.} The key to all of this is to ignore everything which says "All is well".

\end{enumerate}

So you examine the time series and say nothing if its going up and down as you'd expect it to. But, if there's a sudden leap or plummet, something important has happened, and it's crucial that the alerting signal which is called an algedonic is generated. Beer describes algedonics as signals which scream "Ouch it hurts!"

\section*{Cyberfilter}
Cyberfilter is a computer program which takes care of all of this stuff on indicators and algedonics.

You work out your performance indicators at the end of each day and put them into the program. The history of each indicator can be displayed as a graph. It then analyses the graph and tells you if something significant has happened, that is, it generates algedonics.

The program also does several other things involved with short term forecasting and planning, but its main functions from my point of view are:

\begin{itemize}
  \item a simple means of recording performance indicators and

  \item the generation of algedonics.

\end{itemize}

Consider the following crises:

\subsection*{a) The scales break}
Suppose that the scales (which are used to weigh all the pre-packs) get damaged slightly and thus all the packs are 10\% overweight. No-one notices during the day, although there is slight discomfort about the lower yields.

At the end of production, the indicators are worked out and it's clear that wastage has leaped from 2 to 10 \%. In this crisis, everything is checked, the problem is identified and an engineer is called.

(This actually happened a few years ago, but wastage was only monitored infrequently, and no-one really knows how long we were giving away vast amounts of food ... )

\subsection*{b) Change in Personnel}
After some years of efficient production the members of the department feel like a change, and new personnel are chosen. Some training takes place, but after the new personnel are established all the indicators start to slip.

Initially the information goes back only to the pre-packing department, but after a further two weeks no improvement has been made, and the algedonic is sent to the co-operative as a whole and an enquiry is made.

The previous workers are called back in, the situation is sorted out with more training and perhaps some re-allocation of people, and the original performance levels are regenerated. (Again this happens all the time in co-operatives but as there are no performance indicators, the new team has very little basis on which to learn. In some extreme cases, the decline in standards resulted in a department being shut down, but only after the quarterly accounts showed that money had been lost.)

\section*{Summary}
The essence of all this is the design of an information system which:

\begin{itemize}
  \item Underwrites departmental autonomy.

  \item Provides the feed-back to enable a department to learn and adapt.

  \item Ensures that Operational autonomy can only continue as long as they are working to standards set by the needs of the whole organisation.

  \item Only responds with useful information (no more monster computer print-outs to feed the shredder).

  \item Responds to information which is only a few hours out of date.

\end{itemize}

\section*{Example 2: Mondragon}
One of my original problems with the idea of performance indicators was how to deal with performance which is not easy to measure - you can record the number of boxes produced, but how do you measure something like tidiness or morale?

One solution to these problems came from the Fagor refrigerator factory in Mondragon.

They had changed their production line into a series of autonomous work groups in order to address problems of motivation, and had identified performance indicators as the appropriate means of handling information.

As expected, they measured productivity and other quantifiable aspects of their working situation, and this gave them a new degree of autonomy.

For example, during my visit to Mondragon there were a series of festivals which lasted all night and (as expected) the production figures fell dramatically on the following day. This is not seen as a problem as long as the figures rise on subsequent days and the weekly average reaches the usual standard.

All of this is under the control of the people on the shop floor - the algedonic is only generated if the weekly average is affected.

Thus a comprehensive monitoring system can enhance departmental autonomy.

But they were also able to deal with less tangible elements.

They do this by negotiation: Once a week a representative of the work group meets with the foreman and they go through a number of performance indicators. As an example they discuss autonomy which is registered on a scale of 0 to 10. The group is aiming at complete autonomy and may feel it had almost got there: "We think 9 for autonomy". The foreman may disagree "But on Wednesday you couldn't deal with a problem and had to ask me to sort it out for you. I think 6." Eventually they may agree on 8.

All these numbers are written up and displayed on notice boards. They don't plot graphs, but the flow of numbers provides a reasonable picture of the way things are going, and provides the feed back which is one of the more important aspects of this system.

This concept of negotiated performance indicators opens up many possibilities. In a co-operative it seems more likely that the negotiator would be someone who used to work in a particular department, understands how it functions, and has an interest in maintaining its standards.

\section*{How to Design the System}

\section*{1. Performance Indicators}
Negotiations are needed between the department in question and whoever is responsible for the allocation of resources. (Remember this is also designing the accountability systems which complete the resource-bargain loop between Systems 1 and 3).

The question is "What numbers do I regularly quote when I'm talking about how well the day has gone?" For example "Only three tonnes of muesli all afternoon." or "What a good day! I completed four pages of the ledger."

These indicators need to satisfy both the department and the resource allocator that they give a complete picture.

It will then be the responsibility of the department to measure and plot them every day.

\section*{2. Algedonics}
Some variation will be inevitable. It may take some time to establish what an algedonic actually is. So ... 5\% variation in productivity is fine as long as the variations even out. A continuous decline for 4 days is unacceptable and constitutes an algedonic. A 10\% variation needs to be examined. And so on.

If you decide to use Cyberfilter, this kind of decision will still have to be made. The responsiveness of the program has to be established, or it may churn out algedonics every time someone sneezes.

\section*{3. Time Periods.}
Each indicator must be studied individually. You must then decide how long it should take to deal with problems, and how long a problem can be permitted to continue until the viability of the whole co-operative is at risk.

So ... you have 5 days to deal with wastage problems, 10 days to get out-of-stocks back to acceptable levels, and so on.

These time periods must be agreed in advance, as when a crises hits the system, the framework for dealing with it must be already established.

\section*{4. Loss of Autonomy}
Assuming an indicator becomes unacceptable and continues at that level beyond the pre-agreed time, then the whole co-operative gets notified and that department loses its autonomy.

The nature of this loss should again be designed. It may involve a complete analysis of the problem, or the appointment of an agreed trouble-shooter or whatever.

But again this should be agreed in advance.

\section*{Step 12 Examine your Information Systems}
% Box with 3 items numbered 12.1 12.2 12.3
 Consider your current information systems. How do you measure what is happening within each department? How do you ensure that each department is doing the things it is supposed to do? Do you need systems which alert you when something goes wrong, or would it be immediately obvious anyway? How up-to-date is your financial information? If you started to loose money today, how long would it take for you systems to realise?
 
 In the light of the answers to these questions, you should be able to address the crucial issue: How complete and up-to-date is the model of the operation?

If you have qualms about the kind of information systems you currently use, it may be sensible to define and measure performance indicators daily and see how your organisation changes.