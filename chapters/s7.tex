\chapter{BALANCE WITH THE EXTERNAL ENVIRONMENT}
  \label{BALANCE WITH THE EXTERNAL ENVIRONMENT}
In this Section you will be looking at the way that your enterprise deals with changes in the environment in which it exists.

There is no point in having a superbly organised internal environment if the enterprise as a whole is completely out of touch with the environment in which it operates. The rate of change of technological advance continues to accelerate, and businesses cannot afford to stand still and ignore the changes in the market. Currently, computer software designers consider themselves lucky if a particular program leads the market for more than a year.

The focus for these issues is System 4 which is responsible for future planning in the context of environmental information which it gathers.

System 4 must provide the balance between the internal Operational units and the outside world, and must ensure that the organisation can adapt to change.

\section*{Graphical Overview}
The following diagram illustrates the relevant parts of the VSM involved in this Section.

The issues which will be addressed in this Section are:

\begin{itemize}
  \item How does System 4 keep in touch with the outside world?

  \item What capabilities does System 4 need in order to formulate its plans which are needed to ensure that the enterprise can adapt to the changes in the environment.

  \item What is the role of System 3?

  \item What is the role of System 5?

\end{itemize}

\section*{Connection to the External Environment}
System 4 is charged with dealing with future planning in the context of the external environment. Clearly, the first job for System 4 is to decide which part of the (infinite) external environment is of direct relevance.

The VSM distinguishes between two kinds of external environment:

The first is the Predictable which can be monitored. Trends may be identified and decisions made accordingly.

The obvious example of this is the way that a market is changing.

In most businesses it is clear that as market trends alter, the business must adapt. Most large corporations spend enormous amounts of money on Market Research and on running experiments in selected areas to assess the mood of the consumer.

The second is the Novel. Things may be proceeding just as you expect and then someone invents the light bulb. Clearly every Viable System must have some provision for coping with the novel, even if it's only being aware of development programs in the relevant areas.

\section*{Connection to the Internal Environment}
System 4 has to be an integrated part of the Viable System. Its main internal connection is through System 3 which is charged with stabilising and optimising the internal environment.

The intense interaction between Systems 3 and 4 was mentioned in the Preliminary Diagnosis. This is essential as all plans must evolve in the context of both external threats and opportunities, and of the internal capabilities of the System-in-Focus.

The example of a new warehouse was given earlier - System 4 needs to look outwards (at possible sites, financing options, etc. etc.) and to look inwards (necessary square footage, headroom, hygiene standards, etc.) and to make a decision in the light of both sets of information.

In practical terms this means that the need for good communications between Systems 3 and 4 must be recognised.

There is no point in having these two essential aspects of viability working in isolation.

\begin{itemize}
  \item The people making future plans must be in touch with, and take account of, the internal capabilities of the organisation

  \item The people charged with the overview of the internal environment must be aware of the plans being formulated by the future planners.

\end{itemize}

\section*{The Role of System 5}
It's clear that at some stage, decisions will have to be made about the investment to be made in System 3 and System 4.

If the balance is not made correctly disaster may ensue:

Too much emphasis on System 3. Internally, the systems may work wonderfully, but without System 4 the products may become irrelevant, e.g. perfectly designed and manufactured Sedan Chairs.

Too much emphasis on System 4. The future planning may be impeccable but the Inside and Now may be incapable of producing the goods. In this case, the kind of product may be exactly what the market needs but their quality or price may make them unsaleable.

An example of this kind of imbalance is illustrated by Clive Sinclair's production of electronic calculators. His research and development was superb, but the organisation was not viable as his production cost were too high. Too much System 4: not enough emphasis of System 3.

It should be noted that his next venture got the balance right. The ZX computers were well conceived and well produced.

It should also be noted that his Electric Cars displayed his first error in System 4: they were not what the market wanted and despite good quality and prices were (in the UK at least) a complete flop.

The decisions about the investment in Systems 3 and 4 have to be made at the policy level. The decision will have to be made in terms of the nature of the business, and the speed with which the market changes.

This is a job for System 5.

\section*{Monitoring}

\section*{Step 13 Design the Balance with the External Environment}

\section*{Example 1: Small Co-operative}
In a small co-operative most of this is taken care of through the mechanics of thorough discussion.

Systems 3 and 4 are the same people and so the necessary communication is straightforward.

The System 4 plans will be made with thorough knowledge of the capabilities of the Operational units.

Allocation of resources must be performed by deciding to put time aside to look at markets, and to formulate plans.

But, basically the system can work well, as long as the need for these functions is recognised.

\section*{Example 2: Large Co-operative}
The perceived need for a System 4 in large co-operatives seems to vary enormously.

Some co-operatives seem to function without any continuous focus for future planning and with sporadic bursts of activity when the need becomes obvious. Most of these co-operatives function in markets which are not in a rapid state of flux, and have survived without a well defined System 4. Their viability depends upon the markets continuing in this mode, or upon the realisation that continuous future planning is a necessary function.

Other co-operatives recognise the need for System 4, which is often carried out by one of the founder members.

\section*{Example 3: Mondragon}
At every level, the Mondragon co-operatives take System 4 functions very seriously.

They monitor market trends, and technological advances and ensure their production techniques are state-of-the-art. They do their own research and development and from the evidence I had from my visit, they are fully adapted to the external environment.

Currently one of their main preoccupations is with the unified European market, and how it will affect their market position.

It is difficult to assess their allocation of resources without a thorough study, but their seem to put equal emphasis on their future plans and simulations and upon the need for efficient production techniques.
