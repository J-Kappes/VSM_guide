\chapter{DESIGNING POLICY SYSTEMS}
  \label{DESIGNING POLICY SYSTEMS}
From the co-operative point of view, the way that policy is made and implemented is without doubt the most important issue. The business must pay its bills and make a profit, but all of the financial issues are constraints: what the co-operative is really all about is defined partly by its mission statement, and partly by the working environment it creates for its members.

Central to this issue is the involvement that everyone has in making policy.

In a small group this is easy: you have a meeting, discuss an issue until some sort of consensus emerges, and the policy is made.

In a larger group the dynamics of large meetings make this difficult, and choices have to be made about the methods of involving everyone in all the important policy decisions, and how to prevent the business degenerating into endless non-productive meetings.

\section*{System 5: The Design of Policy Systems}
Perhaps the biggest problem at Suma in 1986 was that everyone wanted to be involved in every decision and the resulting proliferation of meetings was getting unworkable. While policy clearly requires an input from every member, it had become necessary to get some decisions made only by the members directly involved.

The VSM contribution to this problem was:

\begin{itemize}
  \item to make departmental decisions within each department.

  \item to appoint officers with an overview to make decisions about the interaction of departments.

  \item to make all of this accountable to all members with clear mechanismsto empower any 5 members to call a General Meeting.

  \item to look at all of this and the policy issues on a weekly basis so that there is no chance of policy slipping away from the control of the membership between infrequent meetings.

\end{itemize}

The situation is, of course, not completely clear. The exact nature of "policy" is indefinable, and it is debatable whether some matters should be decided by an officer or discussed by everyone.

The crucial factors are that:

\begin{itemize}
  \item All officers are accountable.

  \item Mechanisms exist to enable members to change the way that officers are operating.

  \item Clear policy issues cannot be made without the majority of all members agreeing.

\end{itemize}

\section*{Step 14: Design of Policy Structures}

\section*{Example 1}
Suma has been discussed at length. If it may be criticised on its policy systems it must be from the point of view of not letting the officers make decisions on their own. Any slight modification of policy - which could be seen as interpretation of existing policy - has to be discussed by the entire membership.

There is little doubt that the weekly Hub/Sector system keeps everyone involved in all policy matters.

\section*{Example 2}
Other large co-operatives handle decision making in different ways.

They may use a weekly meeting of delegates to take the overview of the current situation, and make decisions on their own initiatives (at Suma, delegates report only their Sectors' views). They then have monthly staff meetings to look at the weekly decisions and to make policy. All weekly decisions and agendas may be pinned on a notice board and members may intervene if they feel it is necessary.

Generally this system seems to work, although there are some grumbles about lack of consultation.
